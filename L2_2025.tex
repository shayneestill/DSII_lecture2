% Options for packages loaded elsewhere
\PassOptionsToPackage{unicode}{hyperref}
\PassOptionsToPackage{hyphens}{url}
%
\documentclass[
]{article}
\usepackage{amsmath,amssymb}
\usepackage{iftex}
\ifPDFTeX
  \usepackage[T1]{fontenc}
  \usepackage[utf8]{inputenc}
  \usepackage{textcomp} % provide euro and other symbols
\else % if luatex or xetex
  \usepackage{unicode-math} % this also loads fontspec
  \defaultfontfeatures{Scale=MatchLowercase}
  \defaultfontfeatures[\rmfamily]{Ligatures=TeX,Scale=1}
\fi
\usepackage{lmodern}
\ifPDFTeX\else
  % xetex/luatex font selection
\fi
% Use upquote if available, for straight quotes in verbatim environments
\IfFileExists{upquote.sty}{\usepackage{upquote}}{}
\IfFileExists{microtype.sty}{% use microtype if available
  \usepackage[]{microtype}
  \UseMicrotypeSet[protrusion]{basicmath} % disable protrusion for tt fonts
}{}
\makeatletter
\@ifundefined{KOMAClassName}{% if non-KOMA class
  \IfFileExists{parskip.sty}{%
    \usepackage{parskip}
  }{% else
    \setlength{\parindent}{0pt}
    \setlength{\parskip}{6pt plus 2pt minus 1pt}}
}{% if KOMA class
  \KOMAoptions{parskip=half}}
\makeatother
\usepackage{xcolor}
\usepackage[margin=1in]{geometry}
\usepackage{color}
\usepackage{fancyvrb}
\newcommand{\VerbBar}{|}
\newcommand{\VERB}{\Verb[commandchars=\\\{\}]}
\DefineVerbatimEnvironment{Highlighting}{Verbatim}{commandchars=\\\{\}}
% Add ',fontsize=\small' for more characters per line
\usepackage{framed}
\definecolor{shadecolor}{RGB}{248,248,248}
\newenvironment{Shaded}{\begin{snugshade}}{\end{snugshade}}
\newcommand{\AlertTok}[1]{\textcolor[rgb]{0.94,0.16,0.16}{#1}}
\newcommand{\AnnotationTok}[1]{\textcolor[rgb]{0.56,0.35,0.01}{\textbf{\textit{#1}}}}
\newcommand{\AttributeTok}[1]{\textcolor[rgb]{0.13,0.29,0.53}{#1}}
\newcommand{\BaseNTok}[1]{\textcolor[rgb]{0.00,0.00,0.81}{#1}}
\newcommand{\BuiltInTok}[1]{#1}
\newcommand{\CharTok}[1]{\textcolor[rgb]{0.31,0.60,0.02}{#1}}
\newcommand{\CommentTok}[1]{\textcolor[rgb]{0.56,0.35,0.01}{\textit{#1}}}
\newcommand{\CommentVarTok}[1]{\textcolor[rgb]{0.56,0.35,0.01}{\textbf{\textit{#1}}}}
\newcommand{\ConstantTok}[1]{\textcolor[rgb]{0.56,0.35,0.01}{#1}}
\newcommand{\ControlFlowTok}[1]{\textcolor[rgb]{0.13,0.29,0.53}{\textbf{#1}}}
\newcommand{\DataTypeTok}[1]{\textcolor[rgb]{0.13,0.29,0.53}{#1}}
\newcommand{\DecValTok}[1]{\textcolor[rgb]{0.00,0.00,0.81}{#1}}
\newcommand{\DocumentationTok}[1]{\textcolor[rgb]{0.56,0.35,0.01}{\textbf{\textit{#1}}}}
\newcommand{\ErrorTok}[1]{\textcolor[rgb]{0.64,0.00,0.00}{\textbf{#1}}}
\newcommand{\ExtensionTok}[1]{#1}
\newcommand{\FloatTok}[1]{\textcolor[rgb]{0.00,0.00,0.81}{#1}}
\newcommand{\FunctionTok}[1]{\textcolor[rgb]{0.13,0.29,0.53}{\textbf{#1}}}
\newcommand{\ImportTok}[1]{#1}
\newcommand{\InformationTok}[1]{\textcolor[rgb]{0.56,0.35,0.01}{\textbf{\textit{#1}}}}
\newcommand{\KeywordTok}[1]{\textcolor[rgb]{0.13,0.29,0.53}{\textbf{#1}}}
\newcommand{\NormalTok}[1]{#1}
\newcommand{\OperatorTok}[1]{\textcolor[rgb]{0.81,0.36,0.00}{\textbf{#1}}}
\newcommand{\OtherTok}[1]{\textcolor[rgb]{0.56,0.35,0.01}{#1}}
\newcommand{\PreprocessorTok}[1]{\textcolor[rgb]{0.56,0.35,0.01}{\textit{#1}}}
\newcommand{\RegionMarkerTok}[1]{#1}
\newcommand{\SpecialCharTok}[1]{\textcolor[rgb]{0.81,0.36,0.00}{\textbf{#1}}}
\newcommand{\SpecialStringTok}[1]{\textcolor[rgb]{0.31,0.60,0.02}{#1}}
\newcommand{\StringTok}[1]{\textcolor[rgb]{0.31,0.60,0.02}{#1}}
\newcommand{\VariableTok}[1]{\textcolor[rgb]{0.00,0.00,0.00}{#1}}
\newcommand{\VerbatimStringTok}[1]{\textcolor[rgb]{0.31,0.60,0.02}{#1}}
\newcommand{\WarningTok}[1]{\textcolor[rgb]{0.56,0.35,0.01}{\textbf{\textit{#1}}}}
\usepackage{graphicx}
\makeatletter
\def\maxwidth{\ifdim\Gin@nat@width>\linewidth\linewidth\else\Gin@nat@width\fi}
\def\maxheight{\ifdim\Gin@nat@height>\textheight\textheight\else\Gin@nat@height\fi}
\makeatother
% Scale images if necessary, so that they will not overflow the page
% margins by default, and it is still possible to overwrite the defaults
% using explicit options in \includegraphics[width, height, ...]{}
\setkeys{Gin}{width=\maxwidth,height=\maxheight,keepaspectratio}
% Set default figure placement to htbp
\makeatletter
\def\fps@figure{htbp}
\makeatother
\setlength{\emergencystretch}{3em} % prevent overfull lines
\providecommand{\tightlist}{%
  \setlength{\itemsep}{0pt}\setlength{\parskip}{0pt}}
\setcounter{secnumdepth}{-\maxdimen} % remove section numbering
\usepackage{fancyhdr}
\usepackage{lipsum}
\pagestyle{fancy}
\fancyhead[R]{\thepage}
\fancypagestyle{plain}{\pagestyle{fancy}}
\ifLuaTeX
  \usepackage{selnolig}  % disable illegal ligatures
\fi
\usepackage{bookmark}
\IfFileExists{xurl.sty}{\usepackage{xurl}}{} % add URL line breaks if available
\urlstyle{same}
\hypersetup{
  pdftitle={An Overview of Modeling Process},
  pdfauthor={Yifei Sun, Runze Cui},
  hidelinks,
  pdfcreator={LaTeX via pandoc}}

\title{An Overview of Modeling Process}
\author{Yifei Sun, Runze Cui}
\date{}

\begin{document}
\maketitle

{
\setcounter{tocdepth}{2}
\tableofcontents
}
\newpage

\begin{Shaded}
\begin{Highlighting}[]
\FunctionTok{library}\NormalTok{(caret)}
\FunctionTok{library}\NormalTok{(tidymodels)}
\FunctionTok{library}\NormalTok{(kknn)}
\FunctionTok{library}\NormalTok{(FNN) }\CommentTok{\# knn.reg()}
\FunctionTok{library}\NormalTok{(doBy) }\CommentTok{\# which.minn()}

\FunctionTok{set.seed}\NormalTok{(}\DecValTok{2025}\NormalTok{)}
\end{Highlighting}
\end{Shaded}

The goal of this tutorial is to provide an overview of the modeling
process. The functions from the packages \texttt{caret} and
\texttt{tidymodels} will be discussed in details in our future lectures.

You can generate a simulated training dataset or use an existing
dataset. For illustration, we use a simulated dataset with two
predictors.

\begin{Shaded}
\begin{Highlighting}[]
\CommentTok{\# Data generating {-} you can replace this with your own function}
\NormalTok{genData }\OtherTok{\textless{}{-}} \ControlFlowTok{function}\NormalTok{(N)}
\NormalTok{\{}
\NormalTok{  X1 }\OtherTok{\textless{}{-}} \FunctionTok{rnorm}\NormalTok{(N, }\AttributeTok{mean =} \DecValTok{1}\NormalTok{)}
\NormalTok{  X2 }\OtherTok{\textless{}{-}} \FunctionTok{rnorm}\NormalTok{(N, }\AttributeTok{mean =} \DecValTok{1}\NormalTok{)}
\NormalTok{  eps }\OtherTok{\textless{}{-}} \FunctionTok{rnorm}\NormalTok{(N, }\AttributeTok{sd =}\NormalTok{ .}\DecValTok{5}\NormalTok{)}
\NormalTok{  Y }\OtherTok{\textless{}{-}} \FunctionTok{sin}\NormalTok{(X1) }\SpecialCharTok{+}\NormalTok{ (X2)}\SpecialCharTok{\^{}}\DecValTok{2} \SpecialCharTok{+}\NormalTok{ eps }
  \CommentTok{\# Y \textless{}{-} X1 + X2 + eps}
  \FunctionTok{data.frame}\NormalTok{(}\AttributeTok{Y =}\NormalTok{ Y, }\AttributeTok{X1 =}\NormalTok{ X1, }\AttributeTok{X2 =}\NormalTok{ X2)}
\NormalTok{\}}

\NormalTok{dat }\OtherTok{\textless{}{-}} \FunctionTok{genData}\NormalTok{(}\DecValTok{500}\NormalTok{)}
\end{Highlighting}
\end{Shaded}

\section{Data partition}\label{data-partition}

\begin{Shaded}
\begin{Highlighting}[]
\NormalTok{datSplit }\OtherTok{\textless{}{-}} \FunctionTok{initial\_split}\NormalTok{(}\AttributeTok{data =}\NormalTok{ dat, }\AttributeTok{prop =} \FloatTok{0.8}\NormalTok{)}
\NormalTok{trainData }\OtherTok{\textless{}{-}} \FunctionTok{training}\NormalTok{(datSplit)}
\NormalTok{testData }\OtherTok{\textless{}{-}} \FunctionTok{testing}\NormalTok{(datSplit)}

\FunctionTok{head}\NormalTok{(trainData)}
\end{Highlighting}
\end{Shaded}

\begin{verbatim}
##          Y       X1        X2
## 1 1.302390 0.803912 0.6353715
## 2 1.997070 1.190459 0.8640040
## 3 8.630112 1.035641 2.9678707
## 4 2.049905 2.152857 1.2522897
## 5 6.250738 1.370975 2.3743564
## 6 8.173025 1.428511 2.6214588
\end{verbatim}

\section{Data visualization}\label{data-visualization}

\subsection{ggplot}\label{ggplot}

\begin{Shaded}
\begin{Highlighting}[]
\NormalTok{p1 }\OtherTok{\textless{}{-}} \FunctionTok{ggplot}\NormalTok{(trainData, }\FunctionTok{aes\_string}\NormalTok{(}\AttributeTok{x =} \StringTok{"X1"}\NormalTok{, }\AttributeTok{y =} \StringTok{"Y"}\NormalTok{)) }\SpecialCharTok{+}
  \FunctionTok{geom\_point}\NormalTok{(}\AttributeTok{color =} \StringTok{"darkgreen"}\NormalTok{, }\AttributeTok{alpha =} \FloatTok{0.5}\NormalTok{) }\SpecialCharTok{+}
  \FunctionTok{geom\_smooth}\NormalTok{(}\AttributeTok{method =} \StringTok{"loess"}\NormalTok{, }\AttributeTok{span =} \FloatTok{0.5}\NormalTok{, }\AttributeTok{color =} \StringTok{"red"}\NormalTok{, }\AttributeTok{se =} \ConstantTok{FALSE}\NormalTok{) }\SpecialCharTok{+}
  \FunctionTok{theme\_bw}\NormalTok{() }\SpecialCharTok{+}
  \FunctionTok{labs}\NormalTok{(}\AttributeTok{x =} \StringTok{"X1"}\NormalTok{, }\AttributeTok{y =} \StringTok{"Y"}\NormalTok{)}
\NormalTok{p2 }\OtherTok{\textless{}{-}} \FunctionTok{ggplot}\NormalTok{(trainData, }\FunctionTok{aes\_string}\NormalTok{(}\AttributeTok{x =} \StringTok{"X2"}\NormalTok{, }\AttributeTok{y =} \StringTok{"Y"}\NormalTok{)) }\SpecialCharTok{+}
  \FunctionTok{geom\_point}\NormalTok{(}\AttributeTok{color =} \StringTok{"darkgreen"}\NormalTok{, }\AttributeTok{alpha =} \FloatTok{0.5}\NormalTok{) }\SpecialCharTok{+}
  \FunctionTok{geom\_smooth}\NormalTok{(}\AttributeTok{method =} \StringTok{"loess"}\NormalTok{, }\AttributeTok{span =} \FloatTok{0.5}\NormalTok{, }\AttributeTok{color =} \StringTok{"red"}\NormalTok{, }\AttributeTok{se =} \ConstantTok{FALSE}\NormalTok{) }\SpecialCharTok{+}
  \FunctionTok{theme\_bw}\NormalTok{() }\SpecialCharTok{+}
  \FunctionTok{labs}\NormalTok{(}\AttributeTok{x =} \StringTok{"X2"}\NormalTok{, }\AttributeTok{y =} \StringTok{"Y"}\NormalTok{)}

\FunctionTok{library}\NormalTok{(patchwork)}
\NormalTok{p1 }\SpecialCharTok{+}\NormalTok{ p2}
\end{Highlighting}
\end{Shaded}

\includegraphics{L2_2025_files/figure-latex/unnamed-chunk-4-1.pdf}

\subsection{featurePlot}\label{featureplot}

\begin{Shaded}
\begin{Highlighting}[]
\NormalTok{theme1 }\OtherTok{\textless{}{-}} \FunctionTok{trellis.par.get}\NormalTok{()}
\NormalTok{theme1}\SpecialCharTok{$}\NormalTok{plot.symbol}\SpecialCharTok{$}\NormalTok{col }\OtherTok{\textless{}{-}} \FunctionTok{rgb}\NormalTok{(.}\DecValTok{2}\NormalTok{, .}\DecValTok{4}\NormalTok{, .}\DecValTok{2}\NormalTok{, .}\DecValTok{5}\NormalTok{)}
\NormalTok{theme1}\SpecialCharTok{$}\NormalTok{plot.symbol}\SpecialCharTok{$}\NormalTok{pch }\OtherTok{\textless{}{-}} \DecValTok{16}
\NormalTok{theme1}\SpecialCharTok{$}\NormalTok{plot.line}\SpecialCharTok{$}\NormalTok{col }\OtherTok{\textless{}{-}} \FunctionTok{rgb}\NormalTok{(.}\DecValTok{8}\NormalTok{, .}\DecValTok{1}\NormalTok{, .}\DecValTok{1}\NormalTok{, }\DecValTok{1}\NormalTok{)}
\NormalTok{theme1}\SpecialCharTok{$}\NormalTok{plot.line}\SpecialCharTok{$}\NormalTok{lwd }\OtherTok{\textless{}{-}} \DecValTok{2}
\NormalTok{theme1}\SpecialCharTok{$}\NormalTok{strip.background}\SpecialCharTok{$}\NormalTok{col }\OtherTok{\textless{}{-}} \FunctionTok{rgb}\NormalTok{(.}\DecValTok{0}\NormalTok{, .}\DecValTok{2}\NormalTok{, .}\DecValTok{6}\NormalTok{, .}\DecValTok{2}\NormalTok{)}
\FunctionTok{trellis.par.set}\NormalTok{(theme1)}

\FunctionTok{featurePlot}\NormalTok{(}\AttributeTok{x =}\NormalTok{ trainData[ ,}\DecValTok{2}\SpecialCharTok{:}\DecValTok{3}\NormalTok{],}
            \AttributeTok{y =}\NormalTok{ trainData[ ,}\DecValTok{1}\NormalTok{],}
            \AttributeTok{plot =} \StringTok{"scatter"}\NormalTok{,}
            \AttributeTok{span =}\NormalTok{ .}\DecValTok{5}\NormalTok{,}
            \AttributeTok{labels =} \FunctionTok{c}\NormalTok{(}\StringTok{"Predictors"}\NormalTok{,}\StringTok{"Y"}\NormalTok{),}
            \AttributeTok{type =} \FunctionTok{c}\NormalTok{(}\StringTok{"p"}\NormalTok{, }\StringTok{"smooth"}\NormalTok{),}
            \AttributeTok{layout =} \FunctionTok{c}\NormalTok{(}\DecValTok{2}\NormalTok{, }\DecValTok{1}\NormalTok{))}
\end{Highlighting}
\end{Shaded}

\includegraphics{L2_2025_files/figure-latex/unnamed-chunk-5-1.pdf}

\section{What is k-Nearest
Neighbour?}\label{what-is-k-nearest-neighbour}

Now let's make prediction for a new data point with \texttt{X1\ =\ 0}
and \texttt{X2\ =\ 0}.

\begin{Shaded}
\begin{Highlighting}[]
\CommentTok{\# scatter plot of X2 vs. X}
\NormalTok{p }\OtherTok{\textless{}{-}} \FunctionTok{ggplot}\NormalTok{(trainData, }\FunctionTok{aes}\NormalTok{(}\AttributeTok{x =}\NormalTok{ X1, }\AttributeTok{y =}\NormalTok{ X2)) }\SpecialCharTok{+} \FunctionTok{geom\_point}\NormalTok{() }\SpecialCharTok{+}
  \FunctionTok{geom\_point}\NormalTok{(}\FunctionTok{aes}\NormalTok{(}\AttributeTok{x =} \DecValTok{0}\NormalTok{, }\AttributeTok{y =} \DecValTok{0}\NormalTok{), }\AttributeTok{colour =} \StringTok{"blue"}\NormalTok{)}
 
\NormalTok{p }
\end{Highlighting}
\end{Shaded}

\includegraphics{L2_2025_files/figure-latex/unnamed-chunk-6-1.pdf}

\subsection{KNN from scratch}\label{knn-from-scratch}

\begin{Shaded}
\begin{Highlighting}[]
\CommentTok{\# find the 5 nearest neighbours of (0,0)}
\NormalTok{dist0 }\OtherTok{\textless{}{-}} \FunctionTok{sqrt}\NormalTok{( (trainData[,}\DecValTok{2}\NormalTok{] }\SpecialCharTok{{-}} \DecValTok{0}\NormalTok{)}\SpecialCharTok{\^{}}\DecValTok{2} \SpecialCharTok{+}\NormalTok{ (trainData[,}\DecValTok{3}\NormalTok{] }\SpecialCharTok{{-}} \DecValTok{0}\NormalTok{)}\SpecialCharTok{\^{}}\DecValTok{2}\NormalTok{ ) }\CommentTok{\# calculate the distances }
\NormalTok{neighbor0 }\OtherTok{\textless{}{-}}\NormalTok{ doBy}\SpecialCharTok{::}\FunctionTok{which.minn}\NormalTok{(dist0, }\AttributeTok{n =} \DecValTok{5}\NormalTok{) }\CommentTok{\# indices of the 5 smallest distances}

\CommentTok{\# visualize the neighbours}
\NormalTok{p }\SpecialCharTok{+} \FunctionTok{geom\_point}\NormalTok{(}\AttributeTok{data =}\NormalTok{ trainData[neighbor0, ], }
               \AttributeTok{colour =} \StringTok{"red"}\NormalTok{)}
\end{Highlighting}
\end{Shaded}

\includegraphics{L2_2025_files/figure-latex/unnamed-chunk-7-1.pdf}

\begin{Shaded}
\begin{Highlighting}[]
\CommentTok{\# calculate the mean outcome of the nearest neighbours as the predicted outcome}
\FunctionTok{mean}\NormalTok{(trainData[neighbor0,}\DecValTok{1}\NormalTok{])}
\end{Highlighting}
\end{Shaded}

\begin{verbatim}
## [1] 0.07314706
\end{verbatim}

\subsection{Using knn.reg()}\label{using-knn.reg}

\begin{Shaded}
\begin{Highlighting}[]
\CommentTok{\# Using the knn.reg() function }
\NormalTok{FNN}\SpecialCharTok{::}\FunctionTok{knn.reg}\NormalTok{(}\AttributeTok{train =}\NormalTok{ trainData[,}\DecValTok{2}\SpecialCharTok{:}\DecValTok{3}\NormalTok{], }
             \AttributeTok{test =} \FunctionTok{c}\NormalTok{(}\DecValTok{0}\NormalTok{,}\DecValTok{0}\NormalTok{), }
             \AttributeTok{y =}\NormalTok{ trainData[,}\DecValTok{1}\NormalTok{], }
             \AttributeTok{k =} \DecValTok{5}\NormalTok{)}
\end{Highlighting}
\end{Shaded}

\begin{verbatim}
## Prediction:
## [1] 0.07314706
\end{verbatim}

\section{\texorpdfstring{Model training in
\texttt{caret}}{Model training in caret}}\label{model-training-in-caret}

We consider two candidate models: KNN and linear regression.

\begin{Shaded}
\begin{Highlighting}[]
\FunctionTok{set.seed}\NormalTok{(}\DecValTok{1}\NormalTok{)}
\NormalTok{fit.knn }\OtherTok{\textless{}{-}} \FunctionTok{train}\NormalTok{(Y }\SpecialCharTok{\textasciitilde{}}\NormalTok{ .,}
                 \AttributeTok{data =}\NormalTok{ trainData,}
                 \AttributeTok{method =} \StringTok{"knn"}\NormalTok{,}
                 \AttributeTok{trControl =} \FunctionTok{trainControl}\NormalTok{(}\AttributeTok{method =} \StringTok{"cv"}\NormalTok{, }\AttributeTok{number =} \DecValTok{10}\NormalTok{), }\CommentTok{\# ten{-}fold cross{-}validation}
                 \AttributeTok{tuneGrid =} \FunctionTok{expand.grid}\NormalTok{(}\AttributeTok{k =} \FunctionTok{seq}\NormalTok{(}\AttributeTok{from =} \DecValTok{1}\NormalTok{, }\AttributeTok{to =} \DecValTok{25}\NormalTok{, }\AttributeTok{by =} \DecValTok{1}\NormalTok{)))}

\FunctionTok{ggplot}\NormalTok{(fit.knn)}
\end{Highlighting}
\end{Shaded}

\includegraphics{L2_2025_files/figure-latex/unnamed-chunk-9-1.pdf}

\begin{Shaded}
\begin{Highlighting}[]
\CommentTok{\# plot(fit.knn)}
\end{Highlighting}
\end{Shaded}

k = 3 was selected.

\begin{Shaded}
\begin{Highlighting}[]
\FunctionTok{set.seed}\NormalTok{(}\DecValTok{1}\NormalTok{)}
\NormalTok{fit.lm }\OtherTok{\textless{}{-}} \FunctionTok{train}\NormalTok{(Y }\SpecialCharTok{\textasciitilde{}}\NormalTok{ .,}
                \AttributeTok{data =}\NormalTok{ trainData,}
                \AttributeTok{method =} \StringTok{"lm"}\NormalTok{,}
                \AttributeTok{trControl =} \FunctionTok{trainControl}\NormalTok{(}\AttributeTok{method =} \StringTok{"cv"}\NormalTok{, }\AttributeTok{number =} \DecValTok{10}\NormalTok{))}
\end{Highlighting}
\end{Shaded}

Which is better?

\begin{Shaded}
\begin{Highlighting}[]
\NormalTok{rs }\OtherTok{\textless{}{-}} \FunctionTok{resamples}\NormalTok{(}\FunctionTok{list}\NormalTok{(}\AttributeTok{knn =}\NormalTok{ fit.knn, }\AttributeTok{lm =}\NormalTok{ fit.lm))}
\FunctionTok{summary}\NormalTok{(rs, }\AttributeTok{metric =} \StringTok{"RMSE"}\NormalTok{)}
\end{Highlighting}
\end{Shaded}

\begin{verbatim}
## 
## Call:
## summary.resamples(object = rs, metric = "RMSE")
## 
## Models: knn, lm 
## Number of resamples: 10 
## 
## RMSE 
##         Min.   1st Qu.    Median      Mean   3rd Qu.     Max. NA's
## knn 0.563449 0.6546565 0.6990729 0.7280618 0.7550321 1.057534    0
## lm  1.072828 1.3087248 1.6505107 1.6355090 1.7614153 2.303038    0
\end{verbatim}

Evaluating the final model on the test data

\begin{Shaded}
\begin{Highlighting}[]
\NormalTok{pred.knn }\OtherTok{\textless{}{-}} \FunctionTok{predict}\NormalTok{(fit.knn, }\AttributeTok{newdata =}\NormalTok{ testData)}
\FunctionTok{RMSE}\NormalTok{(pred.knn, testData[,}\DecValTok{1}\NormalTok{])}
\end{Highlighting}
\end{Shaded}

\begin{verbatim}
## [1] 0.6918917
\end{verbatim}

\begin{Shaded}
\begin{Highlighting}[]
\CommentTok{\# pred.lm \textless{}{-} predict(fit.lm, newdata = testData)}
\CommentTok{\# RMSE(pred.lm, testData[,1])}
\end{Highlighting}
\end{Shaded}

\section{\texorpdfstring{Model training in \texttt{tidymodels}
(optional)}{Model training in tidymodels (optional)}}\label{model-training-in-tidymodels-optional}

We consider two candidate models: KNN and linear regression.

\begin{Shaded}
\begin{Highlighting}[]
\CommentTok{\# Model specification for KNN}
\NormalTok{knn\_spec }\OtherTok{\textless{}{-}} \FunctionTok{nearest\_neighbor}\NormalTok{(}\AttributeTok{neighbors =} \FunctionTok{tune}\NormalTok{()) }\SpecialCharTok{\%\textgreater{}\%}
  \FunctionTok{set\_engine}\NormalTok{(}\StringTok{"kknn"}\NormalTok{) }\SpecialCharTok{\%\textgreater{}\%}
  \FunctionTok{set\_mode}\NormalTok{(}\StringTok{"regression"}\NormalTok{)}

\FunctionTok{set.seed}\NormalTok{(}\DecValTok{1}\NormalTok{)}
\CommentTok{\# Split training data: 10{-}fold cross{-}validation}
\NormalTok{cv\_folds }\OtherTok{\textless{}{-}} \FunctionTok{vfold\_cv}\NormalTok{(trainData, }\AttributeTok{v =} \DecValTok{10}\NormalTok{)}

\CommentTok{\# Set up the workflow}
\NormalTok{knn\_workflow }\OtherTok{\textless{}{-}} \FunctionTok{workflow}\NormalTok{() }\SpecialCharTok{\%\textgreater{}\%}
  \FunctionTok{add\_model}\NormalTok{(knn\_spec) }\SpecialCharTok{\%\textgreater{}\%}
  \FunctionTok{add\_formula}\NormalTok{(Y }\SpecialCharTok{\textasciitilde{}}\NormalTok{ .)}

\CommentTok{\# Specify the grid of k to consider}
\NormalTok{k\_values }\OtherTok{\textless{}{-}} \FunctionTok{tibble}\NormalTok{(}\AttributeTok{neighbors =} \FunctionTok{seq}\NormalTok{(}\AttributeTok{from =} \DecValTok{1}\NormalTok{, }\AttributeTok{to =} \DecValTok{25}\NormalTok{, }\AttributeTok{by =} \DecValTok{1}\NormalTok{))}

\CommentTok{\# Tune the KNN model}
\NormalTok{tune\_knn }\OtherTok{\textless{}{-}} \FunctionTok{tune\_grid}\NormalTok{(}
\NormalTok{  knn\_workflow,}
  \AttributeTok{resamples =}\NormalTok{ cv\_folds,}
  \AttributeTok{grid =}\NormalTok{ k\_values}
\NormalTok{)}

\CommentTok{\# You can autoplot the results to see the performance}
\FunctionTok{autoplot}\NormalTok{(tune\_knn, }\AttributeTok{metric =} \StringTok{"rmse"}\NormalTok{)}
\end{Highlighting}
\end{Shaded}

\includegraphics{L2_2025_files/figure-latex/unnamed-chunk-13-1.pdf}

\begin{Shaded}
\begin{Highlighting}[]
\CommentTok{\# Select the best K value based on a performance metric, e.g., accuracy}
\NormalTok{best\_k }\OtherTok{\textless{}{-}} \FunctionTok{select\_best}\NormalTok{(tune\_knn, }\AttributeTok{metric =} \StringTok{"rmse"}\NormalTok{)}
\end{Highlighting}
\end{Shaded}

k = 5 was selected.

\begin{Shaded}
\begin{Highlighting}[]
\CommentTok{\# Finalize the model with the best K}
\NormalTok{final\_knn\_spec }\OtherTok{\textless{}{-}}\NormalTok{ knn\_spec }\SpecialCharTok{\%\textgreater{}\%}
  \FunctionTok{update}\NormalTok{(}\AttributeTok{neighbors =}\NormalTok{ best\_k}\SpecialCharTok{$}\NormalTok{neighbors)}

\CommentTok{\# Model specification for linear regression}
\NormalTok{lm\_spec }\OtherTok{\textless{}{-}} \FunctionTok{linear\_reg}\NormalTok{() }\SpecialCharTok{\%\textgreater{}\%}
  \FunctionTok{set\_engine}\NormalTok{(}\StringTok{"lm"}\NormalTok{) }\SpecialCharTok{\%\textgreater{}\%}
  \FunctionTok{set\_mode}\NormalTok{(}\StringTok{"regression"}\NormalTok{)}
\end{Highlighting}
\end{Shaded}

Which is better?

\begin{Shaded}
\begin{Highlighting}[]
\FunctionTok{workflow\_set}\NormalTok{(}\AttributeTok{preproc =} \FunctionTok{list}\NormalTok{(Y }\SpecialCharTok{\textasciitilde{}}\NormalTok{ .),}
             \AttributeTok{models =} \FunctionTok{list}\NormalTok{(}\AttributeTok{lm =}\NormalTok{ lm\_spec, }\AttributeTok{knn =}\NormalTok{ final\_knn\_spec)) }\SpecialCharTok{\%\textgreater{}\%}
\FunctionTok{workflow\_map}\NormalTok{(}\AttributeTok{resamples =}\NormalTok{ cv\_folds) }\SpecialCharTok{\%\textgreater{}\%} 
  \FunctionTok{collect\_metrics}\NormalTok{() }\SpecialCharTok{\%\textgreater{}\%}
  \FunctionTok{filter}\NormalTok{(.metric }\SpecialCharTok{==} \StringTok{"rmse"}\NormalTok{) }\SpecialCharTok{\%\textgreater{}\%}
  \FunctionTok{select}\NormalTok{(model, mean)}
\end{Highlighting}
\end{Shaded}

\begin{verbatim}
## # A tibble: 2 x 2
##   model             mean
##   <chr>            <dbl>
## 1 linear_reg       1.64 
## 2 nearest_neighbor 0.721
\end{verbatim}

Evaluating the final model on the test data

\begin{Shaded}
\begin{Highlighting}[]
\FunctionTok{workflow}\NormalTok{() }\SpecialCharTok{\%\textgreater{}\%}
  \FunctionTok{add\_model}\NormalTok{(final\_knn\_spec) }\SpecialCharTok{\%\textgreater{}\%}
  \FunctionTok{add\_formula}\NormalTok{(Y }\SpecialCharTok{\textasciitilde{}}\NormalTok{ .) }\SpecialCharTok{\%\textgreater{}\%}
  \FunctionTok{last\_fit}\NormalTok{(datSplit) }\SpecialCharTok{\%\textgreater{}\%} 
  \FunctionTok{collect\_metrics}\NormalTok{() }\SpecialCharTok{\%\textgreater{}\%}
  \FunctionTok{filter}\NormalTok{(.metric }\SpecialCharTok{==} \StringTok{"rmse"}\NormalTok{)}
\end{Highlighting}
\end{Shaded}

\begin{verbatim}
## # A tibble: 1 x 4
##   .metric .estimator .estimate .config             
##   <chr>   <chr>          <dbl> <chr>               
## 1 rmse    standard       0.666 Preprocessor1_Model1
\end{verbatim}

\begin{Shaded}
\begin{Highlighting}[]
\CommentTok{\# workflow() \%\textgreater{}\%}
\CommentTok{\#   add\_model(lm\_spec) \%\textgreater{}\%}
\CommentTok{\#   add\_formula(Y \textasciitilde{} .) \%\textgreater{}\%}
\CommentTok{\#   last\_fit(datSplit) \%\textgreater{}\% }
\CommentTok{\#   collect\_metrics() \%\textgreater{}\%}
\CommentTok{\#   filter(.metric == "rmse")}
\end{Highlighting}
\end{Shaded}


\end{document}
